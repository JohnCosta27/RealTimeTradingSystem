\documentclass[titlepage]{article}

% Title Page
\title{Real-time Trading Platform}
\author{John Costa}

\begin{document}
\maketitle

\section{Project Description and Motivation}
My project will be a trading platform, that allows users to trade - in real time, with others users on the platform, on various assets. The price of buying and selling is entirely determined by the amount at which users decide to buy and sell assets, very similar to a stock exchange.

My primary inspirations is taken from the great stock exchanges in the world: London Stock Exchange (LSE), and New York Stock Exchange (NYSE), inspiration not because they are innovative ideas, but because of the cheer amount of data that these institutions needs to process every second, and how these institution, hold the entire economy of countries on their shoulders, therefore have to be designed and delivered with immense care.

I am aiming to provide a solution similar to that of a stock exchange, but not restricted to stocks, my platform could trade any asset. Furthermore, I am aiming to deliver a product that is able to be used 24 hours of the day, all year around. To do this, scalability and resilience must be at the core of every single decision I make.

Furthermore, an objective of my project is to provide real-time, blazingly fast service to the users exchanging assets. This is also the hardest part of the entire project, because everything will have to be built to not just handle hundreds of requests a second, but also scale to meet thousands of requests a second, providing a great challenge on my software architecture, as well as my backend design. The way I aim to solve these problems is by having a micro-services architecture, which splits the entire application into smaller services that can each scale individually, as well as each service having their own database. However this presents a challenge of its own, how do the micro-services interact with each other efficiently? I plan on using RabbitMQ as a messaging service between each service, it enables for easy communication between services, supporting various protocols, the one I will attempt to use is AMQP 0-9-1 (Insert citation here), allowing me to create queues between services that are efficiently managed by RabbitMQ.

How does this then relate to Advanced Web Development. The only way that users will be able to interact with the platform is through a web interface, and to allow for such a complicated task, I need to use advanced concepts of web development, such as Single Page Applications (SPA), and multi-threaded processing on the client side using Web Workers or Service works. Furthermore, because the connection between client and server must be real-time I will need to use Web Sockets (WS), which is an application layer protocol for exchanging data through a persistent TCP connection. \\
\\
Summarising then, my core features are:
\begin{itemize}
	\item Buying and selling assets in real-time with other users.
	\item Allow all users to view all on-going and previous trades.
	\item View historical data on all assets (This includes typical candle stick, and line charts)
	\item View assets owned by all users on the platform.
\end{itemize}

\pagebreak

\section{Timeline}
Most of the implementation of the project should be done within term one, with some final testing going over to term two. Like this I can dedicate the majority of my time in the second term to finishing the report and doing final testing on the project. I will further split my development time into two iterations, and attempt to use an agile methodoly with my development.

\subsubsection{Iteration 1}
This iteration will be a basic MVP (Minimal Viable Product), focusing on getting features to work, instead of perfecting the entire product. Here I will do the following:
\begin{itemize}
	\item Backend that allows for real-time trading (no scalability or multi-threading needed in this iteration).
	\item Frontend that provides a basic interface to view and participate in trades
	\item UI to allow users to view their assets
	\item Some DevOps on the backend, using Docker to containarize my services.
\end{itemize}

\subsubsection{Iteration 2}
In the second iteration, I will focus much more on polishing my system and finalising features made in the MVP, it is also about making sure my Backend can scale automatically to as many users as need be. This should take less time than Iteration 1.
\begin{itemize}
	\item UI polishing
	\item Scalability using some docker and (maybe) Kubernetes.
	\item Backend optimisation, making sure I take full advantage of multi-threaded processing.
\end{itemize}

\subsection{Term 1}
This will be where the majority of development work is done. Because a lot of these technologies I am only somewhat comfortable with, there will be a degree of prototyping in early days to make sure that I fully understand the technologies I need to use before making major commitments. I will finish Iteration 1 in this term, and hopefully get the second one started.


\begin{itemize}
	\item Week 1: Complete project plan and finalise core features, research technologies that would be suited for the problem.
	\item Week 2: Prototype with the choose technologies to create a very small MVP which enables users to trade across a Web Socket (WS) connection.
	\item Week 3-8: Iteration 1 and report writing as I go along.
  \item Week 9-10: Testing MVP from Iteration 1 and fixing any issues that come along the way. Also, report writing.
  \item Week 11: Start Iteration 2.
\end{itemize}

\subsection{Term 2}
Term 2 will be more dedicated to finalising the solution and making sure that my solution scales. It will also be about further testing with real users, report writing and hammering any bugs.

\begin{itemize}
	\item Week 1-4: Iteration 2.
  \item Week 5-6: Complete testing of the solution. Black box testing, stress testing and real user tests.
  \item Week 7-11: Spare weeks for report writing and finishing any work that needs finishing.
\end{itemize}

\end{document}
